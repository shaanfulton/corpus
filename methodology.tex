\documentclass[12pt]{article}
\usepackage[utf8]{inputenc}
\usepackage{hyperref}

\title{Methodology}
\author{Shaan Fulton}
\date{\today}

\begin{document}

\maketitle

\section*{Introduction}

The project of a public, digital corpus is grounded upon the belief that comprehension of a concept is most successfully developed in the attempt to explain the concept to another.

This corpus, therefore, is a collection of learnings, ideas, and debates. Anything worth committing to comprehension and memory. While physical notes provide the groundwork for us to sketch out an idea, it is in this corpus that the idea is developed enough to be explain to others. To be placed in the corpus, an idea must be sufficiently fleshed out that we would wish it to be publicly displayed and understood.

\section*{Structure}

Ideas are organized by topic and subject area. Root level subject areas may include:

\begin{itemize}
\item Mathematics
\item Philosophy
\item Computer Science
\item History
\item Inventions
\end{itemize}

Subdirectories are permitted. To link ideas to one another, they are referenced by title.

\end{document}

