\documentclass[12pt]{article}
\usepackage[utf8]{inputenc}
\usepackage{hyperref}

\title{On Retaining Knowledge}
\author{Shaan Fulton}
\date{\today}

\begin{document}

\maketitle

\section*{The Concern}

How is knowledge retained over time? How do you avoid knowledge loss, particularly for complex subjects, if this knowledge is not kept in constant practice and use?

\section*{Learn to Teach}

A concept is no more than a set of neural pathways. If these pathways are constantly exercised, they remain easily accessed. If they are rarely, these connections weaken. However, these pathways are not isolated instances in the mind. The mind is entirely interconnected. Our understanding of linear algebra may be closely tied to an understanding of physics, or concepts in computer science. Although the pathways which define our understanding of linear algebra may overtime diminish, their neighbors might be maintained. To unlock those original pathways, it is simply a matter of reigniting those neighbors in the correct order.

When we learn a concept to the point that we can teach it to another, or more precisely, to ourselves, we develop a ledger of these neighbors. We might explain  a concept in linear algebra with a relevant and well-memorized example in computer science. If we forget this linear algebra concept in the future, returning to this record of our attempt to teach it to ourselves unlocks those neighboring connections, still sufficiently strong to clearly reignite the initial pathway relevant to our forgetten concept.

Thus, learning a concept to the point of teaching it, and then creating a well articulated document of this teaching, is perhaps the most powerful means of maintaining memories long term.

\section*{Considering the Repeated Practice Alternative}

Another method of retaining knowledge is to repeatedly practice all the concepts which we wish to retain.

I believe we can quickly agree upon the impracticality of this when we consider the breadth of knowledge which we wish to retain. With that said, repeated practice is valuable for pieces of knowledge considered critical enough to maintain at an ``arm's length,'' or accessible without reviewing documentation. Outside of these core ideas, the scalability of documented teaching is far more appealing.

\end{document}

