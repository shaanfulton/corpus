\documentclass[12pt]{article}
\usepackage[utf8]{inputenc}
\usepackage{hyperref}
\usepackage{listings}
\usepackage{xcolor}
\usepackage{enumitem}
\usepackage{amsmath}

\title{The Essentials of C Syntax}
\author{Shaan Fulton}
\date{\today}


\definecolor{codegray}{rgb}{0.9,0.9,0.9}
\lstset{
  backgroundcolor=\color{codegray},
  basicstyle=\ttfamily\small,
  breaklines=true,
  frame=single,
  showstringspaces=false,
  columns=flexible
}

\begin{document}
\maketitle
\section*{1. Functions: \texttt{argc} and \texttt{argv}}

A C program starts from \texttt{main}, which can accept command-line arguments:
\begin{lstlisting}
int main(int argc, char *argv[]) {
    // ...
}
\end{lstlisting}
\begin{itemize}[noitemsep]
  \item \texttt{argc}: \# of arguments passed, including program name.
  \item \texttt{argv}: Array of strings (char pointers); each holds one argument.
  \item \textbf{Why:} Enables user input at launch (\texttt{./prog foo bar}).
\end{itemize}

\section*{2. Boolean Values: \texttt{false} vs \texttt{true}}

\begin{itemize}[noitemsep]
  \item In C, \textbf{0} is considered \texttt{false}. Any nonzero value is considered \texttt{true}.
  \item \texttt{NULL} is \textbf{not} a Boolean value. It is a special macro that represents a null pointer (i.e., a pointer that points to nothing, usually defined as \texttt{((void *)0)}).
  \item Booleans are not built-in in classic C. Instead, you can use \texttt{\#include <stdbool.h>} to get the \texttt{bool} type, and the keywords \texttt{true} and \texttt{false} (where \texttt{false} is 0 and \texttt{true} is 1).
\end{itemize}
\begin{lstlisting}
// All of these are "false":
if (0)    // false
if (NULL) // false, since NULL is 0 as a pointer

// These are "true":
if (1)    // true
if (42)   // true

#include <stdbool.h>
bool x = false;  // x is 0
bool y = true;   // y is 1
\end{lstlisting}
\textbf{Key Point:} In C, \texttt{NULL} can only be used in pointer contexts, not as a general Boolean value.

\section*{3. Typed Variables and Memory}

\begin{itemize}[noitemsep]
  \item Every variable has a type, which defines its memory usage:
\end{itemize}
\begin{lstlisting}
int age;        // 4 bytes (usually)
char letter;    // 1 byte
float price;    // 4 bytes
\end{lstlisting}
\begin{itemize}[noitemsep]
  \item Choosing a type = controlling size and range of values.
\end{itemize}

\section*{4. Structs: Custom Types (e.g., \texttt{SONG}, \texttt{PLAYER})}

\begin{lstlisting}
typedef struct {
    char name[50];
    int duration;
} SONG;
SONG s = {"Hello", 210};
\end{lstlisting}
\begin{itemize}[noitemsep]
  \item \texttt{struct}s group variables into one object.
  \item Use for modeling things: songs, players, points, etc.
\end{itemize}

\section*{5. \texttt{if}, \texttt{else}, and Brackets}

\begin{lstlisting}
if (x > 0) {
    // block runs if x > 0
} else {
    // otherwise, this runs
}
\end{lstlisting}
\begin{itemize}[noitemsep]
  \item Always use \{\} for clarity, even for one-line bodies.
\end{itemize}

\section*{6. Loops: \texttt{while} vs \texttt{do...while}}

\begin{lstlisting}
while (condition) {
    // May run zero or more times
}

do {
    // Always runs at least once
} while (condition);
\end{lstlisting}
\begin{itemize}[noitemsep]
  \item \texttt{while}: Checks before running the body.
  \item \texttt{do...while}: Runs body, then checks condition.
\end{itemize}

\section*{7. Switch Statements and \texttt{break}}

\begin{lstlisting}
switch(choice) {
    case 1:
        // code
        break;
    case 2:
        // code
        break;
    default:
        // code
}
\end{lstlisting}
\begin{itemize}[noitemsep]
  \item \texttt{break} exits the switch. Omitting it leads to \textbf{fall-through}, so the other cases just keep running after!
\end{itemize}

\section*{8. Pointers: \texttt{\&} and \texttt{*}}

\begin{itemize}[noitemsep]
  \item \texttt{*} declares a pointer type or dereferences a pointer (accesses what it points to).
  \item \texttt{\&} gives the address of a variable (where it lives in memory).
\end{itemize}

\begin{lstlisting}
int x = 42;
int *p = &x;   // p stores the address of x
*p = 13;       // changes x through the pointer
\end{lstlisting}

\textbf{Key Point:}  
Pointers let you work with memory addresses directly—essential for dynamic memory, arrays, and passing by reference.


\end{document}

