\documentclass[12pt]{article}
\usepackage[utf8]{inputenc}
\usepackage{hyperref}

\title{Hexadecimals as an Effective Shorthand for Binary}
\author{Shaan Fulton}
\date{\today}

\begin{document}

\maketitle

We discuss how a base-2, binary system can be used to represent any form of data in \texttt{representing-data}. Hexadecimals provide a convenient shorthand for representing larger strings of binary data. Hexadecimal is base-16, meaning there are 16 combinations per digit. There are also 16 combinations in a string of four binary digits. Thus we can create a perfect bijection (every input maps to a unique output and every output to a unique input) between one hexadecimal digit and four binary digits. This makes writing binary much cleaner:

\[
\begin{array}{|c|c|}
\hline
\textbf{Hex} & \textbf{Binary (4-bit)} \\
\hline
0 & 0000 \\
1 & 0001 \\
2 & 0010 \\
3 & 0011 \\
4 & 0100 \\
5 & 0101 \\
6 & 0110 \\
7 & 0111 \\
8 & 1000 \\
9 & 1001 \\
A & 1010 \\
B & 1011 \\
C & 1100 \\
D & 1101 \\
E & 1110 \\
F & 1111 \\
\hline
\end{array}
\]

Note we precede hexadecimals with \texttt{0x} and binary base-2 with \texttt{0b}. This is purely a shorthand. Hexadecimals have no real fundamental grounding in computer architecture. When we say \texttt{0xFBC} we're really just saying \texttt{0b111110111100} in a legible way.

Converting between positional number system is rather intuitive. A clear mathematical method is outlined in \texttt{translating-number-systems}.

\end{document}

